\documentclass{article}


% if you need to pass options to natbib, use, e.g.:
%     \PassOptionsToPackage{numbers, compress}{natbib}
% before loading neurips_2024


% ready for submission
% \usepackage[final]{neurips_2024}


% to compile a preprint version, e.g., for submission to arXiv, add add the
% [preprint] option:
    \usepackage[preprint]{neurips_2024}


% to compile a camera-ready version, add the [final] option, e.g.:
%    \usepackage[final]{neurips_2024}


% to avoid loading the natbib package, add option nonatbib:
%    \usepackage[nonatbib]{neurips_2024}


\usepackage[utf8]{inputenc} % allow utf-8 input
\usepackage[T1]{fontenc}    % use 8-bit T1 fonts
\usepackage{hyperref}       % hyperlinks
\usepackage{url}            % simple URL typesetting
\usepackage{booktabs}       % professional-quality tables
\usepackage{amsfonts}       % blackboard math symbols
\usepackage{nicefrac}       % compact symbols for 1/2, etc.
\usepackage{microtype}      % microtypography
\usepackage{xcolor}         % colors


\title{Assignment-1 : AI Boom and \& AI Effect}


% The \author macro works with any number of authors. There are two commands
% used to separate the names and addresses of multiple authors: \And and \AND.
%
% Using \And between authors leaves it to LaTeX to determine where to break the
% lines. Using \AND forces a line break at that point. So, if LaTeX puts 3 of 4
% authors names on the first line, and the last on the second line, try using
% \AND instead of \And before the third author name.


\author{Harshvardhan Patidar\\
  Department of Artificial Intelligence\\
  Indian Institute of Technology Hyderabad\\
  \texttt{ai24btech11015@iith.ac.in}
  % example of co authors
  % \And
  % Coauthor \\
  % Affiliation \\
  % Address \\
}


\begin{document}\



\maketitle



%If you want to add an abstract, use below commands
%\begin{abstract}
%\end{abstract}



%use below command to get heading
%\section{Heading}

%If you don't want it to be included in the index, use
%\section*{}



%use below command to get sub-heading sort of thing
%\subsection{Style}



%use below commands for centering and url accordingly
%\begin{center}
%  \url{http://www.neurips.cc/}
%\end{center}



%Use below commmand for creating new paragraph
%\paragraph{}



%You can use below commands in the text to refer to specific sections (you need to use /label{} to where you are referring 
%\ref{gen_inst}



%use below to have nice tiny inline fractions, to increase space between them, use a tildae as in the latter
%\nicefrac{1}{4} Hello this is harshvardhan, typing his latex assignments
%\nicefrac{1}{4}~ Hello this is harshvardhan, typing his latex assignments



%for adding a footnote (vo jo page ke niche hi niche aate hai)
%\footnote{As in this example.}



%For adding a photo/figure
%\begin{figure}
%  \centering
%  \fbox{\rule[-.5cm]{0cm}{4cm} \rule[-.5cm]{4cm}{0cm}}
%  \caption{Sample figure caption.}
%\end{figure}



%For a table	
%\begin{table}
%  \caption{Sample table title}
%  \label{sample-table}
%  \centering
%  \begin{tabular}{lll}
%    \toprule
%    \multicolumn{2}{c}{Part}                   \\
%    \cmidrule(r){1-2}
%    Name     & Description     & Size ($\mu$m) \\
%    \midrule
%    Dendrite & Input terminal  & $\sim$100     \\
%    Axon     & Output terminal & $\sim$10      \\
%    Soma     & Cell body       & up to $10^6$  \\
%    \bottomrule
%  \end{tabular}
%\end{table}



%use for a giving a vertical space 
%\medskip



%Use for appendix (idk what it is)
%\appendix



%For yes, no or na
%You should answer \answerYes{}, \answerNo{}, or \answerNA{}.
   



\end{document}
