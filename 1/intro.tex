\documentclass{article}


% if you need to pass options to natbib, use, e.g.:
%     \PassOptionsToPackage{numbers, compress}{natbib}
% before loading neurips_2024


% ready for submission
%\usepackage[final]{neurips_2024}


% to compile a preprint version, e.g., for submission to arXiv, add add the
% [preprint] option:
     \usepackage[preprint]{neurips_2024}


% to compile a camera-ready version, add the [final] option, e.g.:
%    \usepackage[final]{neurips_2024}


% to avoid loading the natbib package, add option nonatbib:
%    \usepackage[nonatbib]{neurips_2024}


\usepackage[utf8]{inputenc} % allow utf-8 input
\usepackage[T1]{fontenc}    % use 8-bit T1 fonts
\usepackage{hyperref}       % hyperlinks
\usepackage{url}            % simple URL typesetting
\usepackage{booktabs}       % professional-quality tables
\usepackage{amsfonts}       % blackboard math symbols
\usepackage{nicefrac}       % compact symbols for 1/2, etc.
\usepackage{microtype}      % microtypography
\usepackage{xcolor}         % colors
%\usepackage[backend=biber]{biblatex}
%\addbibresource{references.bib}


\title{Assignment-1 : AI Boom \& AI Effect}


% The \author macro works with any number of authors. There are two commands
% used to separate the names and addresses of multiple authors: \And and \AND.
%
% Using \And between authors leaves it to LaTeX to determine where to break the
% lines. Using \AND forces a line break at that point. So, if LaTeX puts 3 of 4
% authors names on the first line, and the last on the second line, try using
% \AND instead of \And before the third author name.


\author{Harshvardhan Patidar\\
  Department of Artificial Intelligence\\
  Indian Institute of Technology Hyderabad\\
  \texttt{ai24btech11015@iith.ac.in}
  % example of co authors
  % \And
  % Coauthor \\
  % Affiliation \\
  % Address \\
}


\begin{document}\



\maketitle



%If you want to add an abstract, use below commands
%\begin{abstract}
%\end{abstract}



%use below command to get heading
%\section{Heading}

%If you don't want it to be included in the index, use
%\section*{}



%use below command to get sub-heading sort of thing
%\subsection{Style}



%use below commands for centering and url accordingly
%\begin{center}
%  \url{http://www.neurips.cc/}
%\end{center}



%Use below commmand for creating new paragraph
%\paragraph{}



%You can use below commands in the text to refer to specific sections (you need to use /label{} to where you are referring 
%\ref{gen_inst}



%use below to have nice tiny inline fractions, to increase space between them, use a tildae as in the latter
%\nicefrac{1}{4} Hello this is harshvardhan, typing his latex assignments
%\nicefrac{1}{4}~ Hello this is harshvardhan, typing his latex assignments



%for adding a footnote (vo jo page ke niche hi niche aate hai)
%\footnote{As in this example.}



%For adding a photo/figure
%\begin{figure}
%  \centering
%  \fbox{\rule[-.5cm]{0cm}{4cm} \rule[-.5cm]{4cm}{0cm}}
%  \caption{Sample figure caption.}
%\end{figure}



%For a table	
%\begin{table}
%  \caption{Sample table title}
%  \label{sample-table}
%  \centering
%  \begin{tabular}{lll}
%    \toprule
%    \multicolumn{2}{c}{Part}                   \\
%    \cmidrule(r){1-2}
%    Name     & Description     & Size ($\mu$m) \\
%    \midrule
%    Dendrite & Input terminal  & $\sim$100     \\
%    Axon     & Output terminal & $\sim$10      \\
%    Soma     & Cell body       & up to $10^6$  \\
%    \bottomrule
%  \end{tabular}
%\end{table}



%use for a giving a vertical space 
%\medskip



%Use for appendix (idk what it is)
%\appendix



%For yes, no or na
%You should answer \answerYes{}, \answerNo{}, or \answerNA{}.


%\bibliographystyle{plain}
%\bibliography{references}
   


%My code starts here

 \section{AI BOOM}  \cite{ai_boom}

\paragraph{}
	The AI boom, or AI spring, is the continuing period of rapid progress in the field of Artificial Intelligence (AI) starting in the late $2010s$ and gaining international prominence in the $2020s$.
	In $2012$, a research team of University of Toronto used artificial neural networks and deep learning in object recognition for CV, catalyzing the AI boom.
	Apart from this, the advancements in GPUs, amount and quality of training data, generative adversarial networks, diffusion models and transformer architectures.
	By $2022$ Language learning models started being used extensively, in chatbots, text to image and speech synthesis softwares.
	AI is being proposed to be used to advance radical forms of human life extension.
	AI has the potential to be applied to fields of Education, healthcare and transportation.
	The AI boom has also flourished some businesses like NVIDIA, the leading GPU manufacturer.

\paragraph{}
	While there are many positives of the AI boom, like any other thing, it also has its own set of negatives.
	The AI boom also poses threats to some companies which might collapse due to AI, like Google Search.
	Inaccuracy, cybersecurity and intellectual property infringement are the main risks associated with the boom.
	Large LLMs have been criticized for producing text including discriminatory biases related to ethnicity or gender.
	There have been many controversial uses of AI or training of AI which went to the judicial courts as well.
	Deepfake pornography has also become a big issue.
	New laws are being formulated to maintain the decorum of society in presence of such tools.
	AI also contributes to increased power usage as large amounts of electricity is needed to power the GPUs.
	Industry leaders have also warned that human might irreversibly lose control over a sufficiently advanced artificial intelligence model. \\ \\ \\ 




 \section{AI EFFECT} \cite{ai_effect} 

\paragraph{}
	The AI effect, a word coined by John McCarthy, occurs when people stop calling something as AI once it starts working well. 
	They say it's not real thinking as it can be computed. 
	This reforms the goal of AI, and the things which AI has now accomplished are no longer seen as a goal of AI.
	For example, when computers were first used to play chess and solve math problems, it was thought to be a big deal, but slowly people stopped thinking of it in that way and started regarding it as a normal thing and not something that is challenging. 
	Tesler’s Theorem says that, “AI is whatever hasn’t been done yet.”
	The achievements of AI are not being considered as an advancement in AI, but in some other fields.
	Essentially, the achievements of AI become part of regular technology rather than being recognized as advancements in AI.



\paragraph{}
	This effect was also very noticeable during the AI WINTER, when researchers avoided using the term “AI” in order to get more funding and sell their products more. 
	AI is integrated in some way or other into many applications without really being called AI.
	AI is also being tried to be discontinued so that people can have their feeling of being unique and special. 
	Experts in the field of AI think this will keep happening as AI improves, and people will keep changing what they think is real Artificial Intelligence. 
	Because of this, people don’t always realize how much AI is doing, since once AI solves a problem, it’s no longer seen as a challenging and difficult task.


\bibliographystyle{plain}
\bibliography{ref}
  

\end{document}
